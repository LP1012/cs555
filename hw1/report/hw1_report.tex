\documentclass[11pt]{article}
\usepackage[margin=1in]{geometry}
\usepackage{xcolor}
\usepackage{mdframed}
\usepackage{amsmath}
\usepackage{amsfonts}
\usepackage{minted}
\usepackage{booktabs}
\usepackage{graphicx}
\usepackage{subcaption}
\usepackage[noabbrev,capitalize]{cleveref}


% copy cs555_hw_template.tex to cs555_hw_1_1.tex
\newcommand{\yourname}{Liam Pohlmann, Alex Shaffer}
\newcommand{\youremail}{liamp2@illinois.edu, aws9@illinois.edu}
\newcommand{\yournetid}{liamp2, aws9}
\newcommand{\hwnum}{1}
\newcounter{probnum}
\setcounter{probnum}{0}  % start counting so first problem is 1


\newenvironment{problem}[1][Problem]
    {
      \stepcounter{probnum}
      \begin{mdframed}[backgroundcolor=gray!20]
      \textbf{#1 \theprobnum} \\
    }
    {
      \end{mdframed}
    }


\newenvironment{solution}
    {\textit{Solution:}}
    {}

\begin{document}
\noindent
\noindent\rule{\linewidth}{2pt}
\large\textbf{\yourname} \hfill \textbf{CS 555: Homework \# \hwnum}   \\
Email: \youremail \hfill Spring 2026\\
\noindent\rule{\linewidth}{2pt}

\begin{problem}
\textcolor{red}{Convergence of BDF1, BDF2, and Crank-Nicholson.} % TODO
\end{problem}

\begin{solution}
    For both prescribed initial conditions, BFD1 (\cref{fig:p1-bdf1-ic1}, \cref{fig:p1-bdf1-ic2}) and BFD2 (\cref{fig:p1-bdf2-ic1} and \cref{fig:p1-bdf2-ic2}) perform well for a varying number of time steps.
    Additionally, both methods converged to the expected ratio of 2 and 4 for BFD1 and BFD2, respectively (see \cref{tab:bfd1-sine}, \cref{tab:bdf1-cv} and \cref{tab:bfd2-sine}, \cref{tab:bfd2-cv}).
    The Crank-Nicholson method performed well for the sinusoidal initial condition (\cref{fig:p1-cn-ic1}, \cref{tab:cn-sine}) but very poorly for the constant-value initial condition (\cref{fig:p1-cn-ic2}, \cref{tab:cn-cv}).
    The solver appears to struggle resolving the solution on the boundaries, though this problem appeared to lessen as the number of timesteps increased.

    \vspace{0.5cm}
    \noindent
    Code was written in \texttt{Python} and included in this report.
    \clearpage

    \begin{figure}[htbp]
        \centering

        \begin{subfigure}{0.3\textwidth}
            \centering
            \includegraphics[width=\linewidth]{../p1/ic1/p1_bfd1_25.png}
            \caption{$N=25$}
        \end{subfigure}
        \hfill
        \begin{subfigure}{0.3\textwidth}
            \centering
            \includegraphics[width=\linewidth]{../p1/ic1/p1_bfd1_50}
            \caption{$N=50$}
        \end{subfigure}
        \hfill
        \begin{subfigure}{0.3\textwidth}
            \centering
            \includegraphics[width=\linewidth]{../p1/ic1/p1_bfd1_100}
            \caption{$N=100$}
        \end{subfigure}

        \hfill
        \begin{subfigure}{0.3\textwidth}
            \centering
            \includegraphics[width=\linewidth]{../p1/ic1/p1_bfd1_200}
            \caption{$N=200$}
        \end{subfigure}
        \hfill
        \begin{subfigure}{0.3\textwidth}
            \centering
            \includegraphics[width=\linewidth]{../p1/ic1/p1_bfd1_400}
            \caption{$N=400$}
        \end{subfigure}
        \hfill
        \begin{subfigure}{0.3\textwidth}
            \centering
            \includegraphics[width=\linewidth]{../p1/ic1/p1_bfd1_800}
            \caption{$N=800$}
        \end{subfigure}
        \hfill
        \begin{subfigure}{0.3\textwidth}
            \centering
            \includegraphics[width=\linewidth]{../p1/ic1/p1_bfd1_1600}
            \caption{$N=1600$}
        \end{subfigure}

        \caption{BDF1 solution for Problem 1 with sinusoidal initial condition and an increasing number of timesteps.}
        \label{fig:p1-bdf1-ic1}
    \end{figure}


    \begin{table}
        \centering
        \begin{tabular}{rrl}
            \toprule
            \textbf{nt} & \textbf{Relative L2 Error} & \textbf{Ratio}   \\
            \midrule
            25          & 7.6865E-02                 & None             \\
            50          & 3.8698E-02                 & 1.98629517948916 \\
            100         & 1.9415E-02                 & 1.99315253596478 \\
            200         & 9.7243E-03                 & 1.99657867816914 \\
            400         & 4.8663E-03                 & 1.99828978420901 \\
            800         & 2.4342E-03                 & 1.99914584597233 \\
            1600        & 1.2174E-03                 & 1.99957098611054 \\
            \bottomrule
        \end{tabular}
        \caption{BFD1 Sinusoidal Initial Condition Output Table}
        \label{tab:bfd1-sine}
    \end{table}


    \begin{figure}[htbp]
        \centering

        \begin{subfigure}{0.3\textwidth}
            \centering
            \includegraphics[width=\linewidth]{../p1/ic1/p1_bfd2_25.png}
            \caption{$N=25$}
        \end{subfigure}
        \hfill
        \begin{subfigure}{0.3\textwidth}
            \centering
            \includegraphics[width=\linewidth]{../p1/ic1/p1_bfd2_50}
            \caption{$N=50$}
        \end{subfigure}
        \hfill
        \begin{subfigure}{0.3\textwidth}
            \centering
            \includegraphics[width=\linewidth]{../p1/ic1/p1_bfd2_100}
            \caption{$N=100$}
        \end{subfigure}

        \hfill
        \begin{subfigure}{0.3\textwidth}
            \centering
            \includegraphics[width=\linewidth]{../p1/ic1/p1_bfd2_200}
            \caption{$N=200$}
        \end{subfigure}
        \hfill
        \begin{subfigure}{0.3\textwidth}
            \centering
            \includegraphics[width=\linewidth]{../p1/ic1/p1_bfd2_400}
            \caption{$N=400$}
        \end{subfigure}
        \hfill
        \begin{subfigure}{0.3\textwidth}
            \centering
            \includegraphics[width=\linewidth]{../p1/ic1/p1_bfd2_800}
            \caption{$N=800$}
        \end{subfigure}
        \hfill
        \begin{subfigure}{0.3\textwidth}
            \centering
            \includegraphics[width=\linewidth]{../p1/ic1/p1_bfd2_1600}
            \caption{$N=1600$}
        \end{subfigure}

        \caption{BDF2 solution for Problem 1 with sinusoidal initial condition and an increasing number of timesteps.}
        \label{fig:p1-bdf2-ic1}
    \end{figure}


    \begin{table}
        \centering
        \begin{tabular}{rrl}
            \toprule
            \textbf{nt} & \textbf{Relative L2 Error} & \textbf{Ratio}   \\
            \midrule
            25          & 5.9969E-04                 & None             \\
            50          & 1.4517E-04                 & 4.13089413461047 \\
            100         & 3.5991E-05                 & 4.0335432954678  \\
            200         & 8.9772E-06                 & 4.00916835052149 \\
            400         & 2.2414E-06                 & 4.00523917104543 \\
            800         & 5.6085E-07                 & 3.9963662517768  \\
            1600        & 1.3945E-07                 & 4.02176317200121 \\
            \bottomrule
        \end{tabular}
        \caption{BFD2 Sinusoidal Initial Condition Output Table}
        \label{tab:bfd2-sine}
    \end{table}


    \begin{figure}[htbp]
        \centering

        \begin{subfigure}{0.3\textwidth}
            \centering
            \includegraphics[width=\linewidth]{../p1/ic1/p1_cn_25.png}
            \caption{$N=25$}
        \end{subfigure}
        \hfill
        \begin{subfigure}{0.3\textwidth}
            \centering
            \includegraphics[width=\linewidth]{../p1/ic1/p1_cn_50}
            \caption{$N=50$}
        \end{subfigure}
        \hfill
        \begin{subfigure}{0.3\textwidth}
            \centering
            \includegraphics[width=\linewidth]{../p1/ic1/p1_cn_100}
            \caption{$N=100$}
        \end{subfigure}

        \hfill
        \begin{subfigure}{0.3\textwidth}
            \centering
            \includegraphics[width=\linewidth]{../p1/ic1/p1_cn_200}
            \caption{$N=200$}
        \end{subfigure}
        \hfill
        \begin{subfigure}{0.3\textwidth}
            \centering
            \includegraphics[width=\linewidth]{../p1/ic1/p1_cn_400}
            \caption{$N=400$}
        \end{subfigure}
        \hfill
        \begin{subfigure}{0.3\textwidth}
            \centering
            \includegraphics[width=\linewidth]{../p1/ic1/p1_cn_800}
            \caption{$N=800$}
        \end{subfigure}
        \hfill
        \begin{subfigure}{0.3\textwidth}
            \centering
            \includegraphics[width=\linewidth]{../p1/ic1/p1_cn_1600}
            \caption{$N=1600$}
        \end{subfigure}

        \caption{Crank-Nicholson solution for Problem 1 with sinusoidal initial condition and an increasing number of timesteps.}
        \label{fig:p1-cn-ic1}
    \end{figure}


    \begin{table}
        \centering
        \begin{tabular}{rrl}
            \toprule
            \textbf{nt} & \textbf{Relative L2 Error} & \textbf{Ratio}   \\
            \midrule
            25          & 1.0259E-03                 & None             \\
            50          & 2.5639E-04                 & 4.00132896073532 \\
            100         & 6.4090E-05                 & 4.00049420088793 \\
            200         & 1.6019E-05                 & 4.0008429926872  \\
            400         & 4.0023E-06                 & 4.00243566210429 \\
            800         & 9.9742E-07                 & 4.01269685473481 \\
            1600        & 2.4586E-07                 & 4.05692780622427 \\
            \bottomrule
        \end{tabular}
        \caption{Crank-Nicholson Sinusoidal Initial Condition Output Table}
        \label{tab:cn-sine}
    \end{table}



    \begin{figure}[htbp]
        \centering

        \begin{subfigure}{0.3\textwidth}
            \centering
            \includegraphics[width=\linewidth]{../p1/ic2/p1_bfd1_25.png}
            \caption{$N=25$}
        \end{subfigure}
        \hfill
        \begin{subfigure}{0.3\textwidth}
            \centering
            \includegraphics[width=\linewidth]{../p1/ic2/p1_bfd1_50}
            \caption{$N=50$}
        \end{subfigure}
        \hfill
        \begin{subfigure}{0.3\textwidth}
            \centering
            \includegraphics[width=\linewidth]{../p1/ic2/p1_bfd1_100}
            \caption{$N=100$}
        \end{subfigure}

        \hfill
        \begin{subfigure}{0.3\textwidth}
            \centering
            \includegraphics[width=\linewidth]{../p1/ic2/p1_bfd1_200}
            \caption{$N=200$}
        \end{subfigure}
        \hfill
        \begin{subfigure}{0.3\textwidth}
            \centering
            \includegraphics[width=\linewidth]{../p1/ic2/p1_bfd1_400}
            \caption{$N=400$}
        \end{subfigure}
        \hfill
        \begin{subfigure}{0.3\textwidth}
            \centering
            \includegraphics[width=\linewidth]{../p1/ic2/p1_bfd1_800}
            \caption{$N=800$}
        \end{subfigure}
        \hfill
        \begin{subfigure}{0.3\textwidth}
            \centering
            \includegraphics[width=\linewidth]{../p1/ic2/p1_bfd1_1600}
            \caption{$N=1600$}
        \end{subfigure}

        \caption{BDF1 solution for Problem 1 with constant-value initial condition and an increasing number of timesteps.}
        \label{fig:p1-bdf1-ic2}
    \end{figure}


    \begin{table}
        \centering
        \begin{tabular}{rrl}
            \toprule
            \textbf{nt} & \textbf{Relative L2 Error} & \textbf{Ratio}   \\
            \midrule
            25          & 7.6865E-02                 & None             \\
            50          & 3.8698E-02                 & 1.98629527605244 \\
            100         & 1.9415E-02                 & 1.99315272312291 \\
            200         & 9.7243E-03                 & 1.9965790456925  \\
            400         & 4.8663E-03                 & 1.99829052801913 \\
            800         & 2.4342E-03                 & 1.99914733965774 \\
            1600        & 1.2174E-03                 & 1.99957398870672 \\
            \bottomrule
        \end{tabular}
        \caption{BFD1 Constant-Value Initial Condition Output Table}
        \label{tab:bdf1-cv}
    \end{table}


    \begin{figure}[htbp]
        \centering

        \begin{subfigure}{0.3\textwidth}
            \centering
            \includegraphics[width=\linewidth]{../p1/ic2/p1_bfd2_25.png}
            \caption{$N=25$}
        \end{subfigure}
        \hfill
        \begin{subfigure}{0.3\textwidth}
            \centering
            \includegraphics[width=\linewidth]{../p1/ic2/p1_bfd2_50}
            \caption{$N=50$}
        \end{subfigure}
        \hfill
        \begin{subfigure}{0.3\textwidth}
            \centering
            \includegraphics[width=\linewidth]{../p1/ic2/p1_bfd2_100}
            \caption{$N=100$}
        \end{subfigure}

        \hfill
        \begin{subfigure}{0.3\textwidth}
            \centering
            \includegraphics[width=\linewidth]{../p1/ic2/p1_bfd2_200}
            \caption{$N=200$}
        \end{subfigure}
        \hfill
        \begin{subfigure}{0.3\textwidth}
            \centering
            \includegraphics[width=\linewidth]{../p1/ic2/p1_bfd2_400}
            \caption{$N=400$}
        \end{subfigure}
        \hfill
        \begin{subfigure}{0.3\textwidth}
            \centering
            \includegraphics[width=\linewidth]{../p1/ic2/p1_bfd2_800}
            \caption{$N=800$}
        \end{subfigure}
        \hfill
        \begin{subfigure}{0.3\textwidth}
            \centering
            \includegraphics[width=\linewidth]{../p1/ic2/p1_bfd2_1600}
            \caption{$N=1600$}
        \end{subfigure}

        \caption{BDF2 solution for Problem 1 with constant-value initial condition and an increasing number of timesteps.}
        \label{fig:p1-bdf2-ic2}
    \end{figure}


    \begin{table}
        \centering
        \begin{tabular}{rrl}
            \toprule
            \textbf{nt} & \textbf{Relative L2 Error} & \textbf{Ratio}   \\
            \midrule
            25          & 5.9968E-04                 & None             \\
            50          & 1.4517E-04                 & 4.13097377163643 \\
            100         & 3.5987E-05                 & 4.03384761961328 \\
            200         & 8.9736E-06                 & 4.01033601077211 \\
            400         & 2.2382E-06                 & 4.00931897266113 \\
            800         & 5.5911E-07                 & 4.00312492433322 \\
            1600        & 1.4345E-07                 & 3.89762898733195 \\
            \bottomrule
        \end{tabular}
        \caption{BFD2 Constant-Value Initial Condition Output Table}
        \label{tab:bfd2-cv}
    \end{table}


    \begin{figure}[htbp]
        \centering

        \begin{subfigure}{0.3\textwidth}
            \centering
            \includegraphics[width=\linewidth]{../p1/ic2/p1_cn_25.png}
            \caption{$N=25$}
        \end{subfigure}
        \hfill
        \begin{subfigure}{0.3\textwidth}
            \centering
            \includegraphics[width=\linewidth]{../p1/ic2/p1_cn_50}
            \caption{$N=50$}
        \end{subfigure}
        \hfill
        \begin{subfigure}{0.3\textwidth}
            \centering
            \includegraphics[width=\linewidth]{../p1/ic2/p1_cn_100}
            \caption{$N=100$}
        \end{subfigure}

        \hfill
        \begin{subfigure}{0.3\textwidth}
            \centering
            \includegraphics[width=\linewidth]{../p1/ic2/p1_cn_200}
            \caption{$N=200$}
        \end{subfigure}
        \hfill
        \begin{subfigure}{0.3\textwidth}
            \centering
            \includegraphics[width=\linewidth]{../p1/ic2/p1_cn_400}
            \caption{$N=400$}
        \end{subfigure}
        \hfill
        \begin{subfigure}{0.3\textwidth}
            \centering
            \includegraphics[width=\linewidth]{../p1/ic2/p1_cn_800}
            \caption{$N=800$}
        \end{subfigure}
        \hfill
        \begin{subfigure}{0.3\textwidth}
            \centering
            \includegraphics[width=\linewidth]{../p1/ic2/p1_cn_1600}
            \caption{$N=1600$}
        \end{subfigure}

        \caption{Crank-Nicholson solution for Problem 1 with constant-value initial condition and an increasing number of timesteps.}
        \label{fig:p1-cn-ic2}
    \end{figure}


    \begin{table}
        \centering
        \begin{tabular}{rrl}
            \toprule
            \textbf{nt} & \textbf{Relative L2 Error} & \textbf{Ratio}   \\
            \midrule
            25          & 6.7230E-01                 & None             \\
            50          & 4.7317E-01                 & 1.42083681543755 \\
            100         & 3.3144E-01                 & 1.42761547073308 \\
            200         & 2.2994E-01                 & 1.44143618794027 \\
            400         & 1.5639E-01                 & 1.47032932084702 \\
            800         & 1.0198E-01                 & 1.53353211016495 \\
            1600        & 6.0521E-02                 & 1.68499598728917 \\
            \bottomrule
        \end{tabular}
        \caption{Crank-Nicholson Constant-Value Initial Condition Output Table}
        \label{tab:cn-cv}
    \end{table}


    \clearpage
    \inputminted[linenos]{python}{../p1/p1.py}
    \clearpage

\end{solution}

\begin{problem}
\textcolor{red}{Growth Factors} % TODO
\end{problem}


\begin{solution}
    In part (a) we are asked to derive the asymptotic growth factor $G(\lambda \Delta t)$ for CN given that $\lambda \in \mathbb{R}$.
    The formula for CN is:

    \[
        u^{n+1} = u^n + \frac{\Delta t}{2} \lambda (u^n + u^{n+1})
    \]

    We look for the growth factor $G$ such that $u^{n+1} = G u^n$.

    \begin{align}
        u^{n+1}                                             & = u^n + \frac{\Delta t}{2} \lambda (u^n + u^{n+1})                      \\
        u^{n+1} - \frac{\Delta t}{2} \lambda u^{n+1}        & = u^n + \frac{\Delta t}{2} \lambda u^n                                  \\
        u^{n+1} \left(1 - \frac{\Delta t}{2} \lambda\right) & = u^n \left(1 + \frac{\Delta t}{2} \lambda\right)                       \\
        G                                                   & = \frac{1 + \frac{\Delta t}{2} \lambda}{1 - \frac{\Delta t}{2} \lambda}
    \end{align}

    As $\lambda \Delta t \to \infty$, we get:
    \[
        \lim_{\lambda \Delta t \to \infty} G = \lim_{\lambda \Delta t \to \infty} \frac{1 + \frac{\Delta t}{2} \lambda}{1 - \frac{\Delta t}{2} \lambda} = -1
    \]

    In part (b) we are asked to plot the growth factors of the analytic solution, forward Euler, backward Euler, and CN for $\lambda \Delta t \in [-10, 0]$.

    \begin{figure}[htbp]
        \centering
        \includegraphics[width=0.8\textwidth]{../p2/growth_factors.png}
        \caption{Growth factors for analytic solution, forward Euler, backward Euler, and Crank-Nicholson methods.}
        \label{fig:growth-factors}
    \end{figure}

\end{solution}

\begin{problem}
\textcolor{red}{\texttt{symamd} speedup.} % TODO
\end{problem}

\begin{solution}
    (i) The original code gives the output seen in Table~\ref{tab:p3-original} with a runtime of 33.164 seconds.

    \begin{table}[htb!]
        \centering
        \begin{tabular}{rrrrrr}
            \toprule
            \textbf{N} & \textbf{$\Delta t$} & \textbf{$n_\text{time steps}$} & \textbf{time} & \textbf{$L2$ Error Ratio} & \textbf{Ratio} \\
            \midrule
            150        & 1.50E-01            & 8                              & 1.20          & 2.5530E-01                & 3.9170E+00     \\
            150        & 8.00E-02            & 15                             & 1.20          & 7.5673E-02                & 3.3737E+00     \\
            150        & 4.00E-02            & 30                             & 1.20          & 1.9138E-02                & 3.9540E+00     \\
            150        & 2.00E-02            & 60                             & 1.20          & 4.7981E-03                & 3.9887E+00     \\
            150        & 1.00E-02            & 120                            & 1.20          & 1.2004E-03                & 3.9972E+00     \\
            150        & 5.00E-03            & 240                            & 1.20          & 3.0015E-04                & 3.9993E+00     \\
            150        & 2.50E-03            & 480                            & 1.20          & 7.5040E-05                & 3.9998E+00     \\
            150        & 1.25E-03            & 960                            & 1.20          & 1.8760E-05                & 4.0000E+00     \\
            \bottomrule
        \end{tabular}
        \caption{Given code output.}
        \label{tab:p3-original}
    \end{table}


    Implemented code (only showing changed solver section):
    \begin{minted}{octave}
        p = symamd(HL);
        L = chol(HL(p,p),'lower');
        HRp = HR(p,p);
        for istep=1:nstep; time=istep*dt;
            rhs = HRp*u(p);
            u = L'\( L \ (rhs));
            u(p) = u;
        end;
    \end{minted}
    This creates the table seen in \cref{tab:part-1}, which, as follows from the logic presented in the problem description, is the exact same.
    The runtime was estimated to be 5.2506 seconds.
    Compared to an initial runtime of 44.548, this represents about an 8.5x performance improvement.

    \begin{table}[htb!]
        \centering
        \begin{tabular}{rrrrrr}
            \toprule
            \textbf{N} & \textbf{$\Delta t$} & \textbf{$n_\text{time steps}$} & \textbf{time} & \textbf{$L2$ Error Ratio} & \textbf{Ratio} \\
            \midrule
            150        & 1.50E-01            & 8                              & 1.20          & 2.5530E-01                & 3.9170E+00     \\
            150        & 8.00E-02            & 15                             & 1.20          & 7.5673E-02                & 3.3737E+00     \\
            150        & 4.00E-02            & 30                             & 1.20          & 1.9138E-02                & 3.9540E+00     \\
            150        & 2.00E-02            & 60                             & 1.20          & 4.7981E-03                & 3.9887E+00     \\
            150        & 1.00E-02            & 120                            & 1.20          & 1.2004E-03                & 3.9972E+00     \\
            150        & 5.00E-03            & 240                            & 1.20          & 3.0015E-04                & 3.9993E+00     \\
            150        & 2.50E-03            & 480                            & 1.20          & 7.5040E-05                & 3.9998E+00     \\
            150        & 1.25E-03            & 960                            & 1.20          & 1.8760E-05                & 4.0000E+00     \\
            \bottomrule
        \end{tabular}
        \caption{Part 1 Results}
        \label{tab:part-1}
    \end{table}

    The number of spatial points was then varied and compared with the original code.
    This experiment yielded the chart seen in \cref{fig:vary-n-comparison}.
    As can be seen, the computation time for the given code grows much more rapidly than that of the modified code.

    \begin{figure}
        \centering
        \includegraphics[width=0.9\textwidth]{../p3/part_1/timings_chart.png}
        \caption{Comparison of improved code in problem 3(i) with given MatLab code. Number of spatial points, $N$, was varied from 75 to 300 in increments of 75. Runtime is report in seconds.}
        \label{fig:vary-n-comparison}
    \end{figure}

    (ii)
    Implementing the ADI algorithm saw an improvement in the runtime. We got a runtime of 0.4397 seconds as opposed the 5.2506 seconds from part (i).
    The matlab code provided fixes N, but varies $\Delta t$. If we vary N instead, we get the runtime scaling that we expect with ADI of $O(n_x n_y) = O(N^2)$.
    For the error scaling, both spatial and temporal errors contribute. Being careful to avoid saturation from one or the other, we can achieve the quadratic error scaling we expect.

    \begin{table}[htb!]
        \centering
        \begin{tabular}{rrrrrr}
            \toprule
            \textbf{N} & \textbf{$\Delta t$} & \textbf{$n_\text{time steps}$} & \textbf{time} & \textbf{$L2$ Error Ratio} & \textbf{Ratio} \\
            \midrule
            150        & 1.50E-01            & 8                              & 1.20          & 1.9057E-01                & 5.2473E+00     \\
            150        & 8.00E-02            & 15                             & 1.20          & 5.5571E-02                & 3.4294E+00     \\
            150        & 4.00E-02            & 30                             & 1.20          & 1.3990E-02                & 3.9721E+00     \\
            150        & 2.00E-02            & 60                             & 1.20          & 3.5036E-03                & 3.9931E+00     \\
            150        & 1.00E-02            & 120                            & 1.20          & 8.7628E-04                & 3.9983E+00     \\
            150        & 5.00E-03            & 240                            & 1.20          & 2.1909E-04                & 3.9996E+00     \\
            150        & 2.50E-03            & 480                            & 1.20          & 5.4775E-05                & 3.9999E+00     \\
            150        & 1.25E-03            & 960                            & 1.20          & 1.3694E-05                & 4.0000E+00     \\
            \bottomrule
        \end{tabular}
        \caption{Part 2 Results}
        \label{tab:part-2}
    \end{table}



\end{solution}

\end{document}
