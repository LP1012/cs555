\documentclass[11pt]{article}
\usepackage[margin=1in]{geometry}
\usepackage{xcolor}
\usepackage{mdframed}
\usepackage{amsmath}
\usepackage{minted}
\usepackage{booktabs}
\usepackage{graphicx}
\usepackage{subcaption}


% copy cs555_hw_template.tex to cs555_hw_1_1.tex
\newcommand{\yourname}{Liam Pohlmann, Alex Shaffer}
\newcommand{\youremail}{liamp2@illinois.edu, aws9@illinois.edu}
\newcommand{\yournetid}{liamp2, aws9}
\newcommand{\hwnum}{1}
\newcounter{probnum}
\setcounter{probnum}{0}  % start counting so first problem is 1


\newenvironment{problem}[1][Problem]
    {
      \stepcounter{probnum}
      \begin{mdframed}[backgroundcolor=gray!20]
      \textbf{#1 \theprobnum} \\
    }
    {
      \end{mdframed}
    }


\newenvironment{solution}
    {\textit{Solution:}}
    {}

\begin{document}
\noindent
\noindent\rule{\linewidth}{2pt}
\large\textbf{\yourname} \hfill \textbf{CS 555: Homework \# \hwnum}   \\
Email: \youremail \hfill Spring 2026\\
\noindent\rule{\linewidth}{2pt}

\begin{problem}
\textcolor{red}{Convergence of BDF1, BDF2, and Crank-Nicholson.} % TODO
\end{problem}

\begin{solution}
    \begin{figure}[htbp]
        \centering

        \begin{subfigure}{0.45\textwidth}
            \centering
            \includegraphics[width=\linewidth]{../p1/ic1/p1_bfd2_25.png}
            \caption{$N=25$}
        \end{subfigure}
        \hfill
        \begin{subfigure}{0.45\textwidth}
            \centering
            \includegraphics[width=\linewidth]{../p1/ic1/p1_bfd2_50}
            \caption{$N=50$}
        \end{subfigure}

        \medskip

        \begin{subfigure}{0.45\textwidth}
            \centering
            \includegraphics[width=\linewidth]{../p1/ic1/p1_bfd2_100}
            \caption{$N=100$}
        \end{subfigure}
        \hfill
        \begin{subfigure}{0.45\textwidth}
            \centering
            \includegraphics[width=\linewidth]{../p1/ic1/p1_bfd2_200}
            \caption{$N=200$}
        \end{subfigure}

        \medskip

        \begin{subfigure}{0.45\textwidth}
            \centering
            \includegraphics[width=\linewidth]{../p1/ic1/p1_bfd2_400}
            \caption{$N=400$}
        \end{subfigure}
        \hfill
        \begin{subfigure}{0.45\textwidth}
            \centering
            \includegraphics[width=\linewidth]{../p1/ic1/p1_bfd2_800}
            \caption{$N=800$}
        \end{subfigure}

        \medskip

        \begin{subfigure}{0.6\textwidth}
            \centering
            \includegraphics[width=\linewidth]{../p1/ic1/p1_bfd2_1600}
            \caption{$N=1600$}
        \end{subfigure}

        \caption{BDF2 solution for Problem 1 with sinusoidal initial condition and an increasing number of timesteps.}
        \label{fig:p1_bdf2_convergence}
    \end{figure}

\end{solution}

\begin{problem}
\textcolor{red}{DESCRIPTION HERE} % TODO
\end{problem}


\begin{solution}
    % TODO
\end{solution}

\begin{problem}
\textcolor{red}{\texttt{symamd} speedup.} % TODO
\end{problem}

\begin{solution}
    The original code gives the output seen in Table~\ref{tab:p3-original} with a runtime of 33.164 seconds.

    \begin{table}[htb!]
        \centering
        \begin{tabular}{rrrrrr}
            \toprule
            \textbf{N} & \textbf{$\Delta t$} & \textbf{$n_\text{time steps}$} & \textbf{time} & \textbf{$L2$ Error Ratio} & \textbf{Ratio} \\
            \midrule
            150        & 1.50E-01            & 8                              & 1.20          & 2.5530E-01                & 3.9170E+00     \\
            150        & 8.00E-02            & 15                             & 1.20          & 7.5673E-02                & 3.3737E+00     \\
            150        & 4.00E-02            & 30                             & 1.20          & 1.9138E-02                & 3.9540E+00     \\
            150        & 2.00E-02            & 60                             & 1.20          & 4.7981E-03                & 3.9887E+00     \\
            150        & 1.00E-02            & 120                            & 1.20          & 1.2004E-03                & 3.9972E+00     \\
            150        & 5.00E-03            & 240                            & 1.20          & 3.0015E-04                & 3.9993E+00     \\
            150        & 2.50E-03            & 480                            & 1.20          & 7.5040E-05                & 3.9998E+00     \\
            150        & 1.25E-03            & 960                            & 1.20          & 1.8760E-05                & 4.0000E+00     \\
            \bottomrule
        \end{tabular}
        \caption{Given code output.}
        \label{tab:p3-original}
    \end{table}


    Implemented code (only showing changed solver section):
    \begin{minted}{octave}
        p = symamd(HL);
        L = chol(HL(p,p),'lower');
        HRp = HR(p,p);
        for istep=1:nstep; time=istep*dt;
            rhs = HRp*u(p);
            u = L'\( L \ (rhs));
            u(p) = u;
        end;
    \end{minted}
    This creates the table seen in Table~\ref{tab:part-1}, which, as follows from the logic presented in the problem description, is the exact same.
    The runtime was estimated to be 3.9840 seconds.
    Compared to an initial runtime of 33.164, this represents about an 8.3x performance improvement.

    \begin{table}[htb!]
        \centering
        \begin{tabular}{rrrrrr}
            \toprule
            \textbf{N} & \textbf{$\Delta t$} & \textbf{$n_\text{time steps}$} & \textbf{time} & \textbf{$L2$ Error Ratio} & \textbf{Ratio} \\
            \midrule
            150        & 1.50E-01            & 8                              & 1.20          & 2.5530E-01                & 3.9170E+00     \\
            150        & 8.00E-02            & 15                             & 1.20          & 7.5673E-02                & 3.3737E+00     \\
            150        & 4.00E-02            & 30                             & 1.20          & 1.9138E-02                & 3.9540E+00     \\
            150        & 2.00E-02            & 60                             & 1.20          & 4.7981E-03                & 3.9887E+00     \\
            150        & 1.00E-02            & 120                            & 1.20          & 1.2004E-03                & 3.9972E+00     \\
            150        & 5.00E-03            & 240                            & 1.20          & 3.0015E-04                & 3.9993E+00     \\
            150        & 2.50E-03            & 480                            & 1.20          & 7.5040E-05                & 3.9998E+00     \\
            150        & 1.25E-03            & 960                            & 1.20          & 1.8760E-05                & 4.0000E+00     \\
            \bottomrule
        \end{tabular}
        \caption{Part 1 Results}
        \label{tab:part-1}
    \end{table}



\end{solution}

\end{document}
